\section {Operatoren}

\subsection{Arten von Operatoren}

%\newcommand{\nov}[1]{\alert{#1}}
\newcommand{\cppop}[1]{ \texttt{#1} }
\newcommand{\pppp}{}


\begin{frame}[fragile]{Alle Operatoren in C++}
	\begin{block}{Alle Operatoren in C++}
		\begin{tabular}{r|l}
			memory			&	\verb|* & new new[] delete delete[]| \verb|sizeof|					\pause \\
			arithmetic		&	\verb!+ - * / % ^ & | ~ << >>!									\pause \\
			logic			&	\verb!&& ||! \verb|!|											\pause \\
			comparison		&	\verb|< > <= >= == !=|											\pause \\
			assignment		&	\verb!= += -= *= /= %= ^= &= |= >>= <<= ++ --!					\pause \\
			others			&	\verb|() [] ,|~~\verb|::|~~\verb|?:| \verb|typeid|				\pause \\
			member access	&	\verb|->* ->|~~\verb|.|~~\verb|.*|								\pause \\
		\end{tabular}
	\end{block}
	
	\vspace{2em}
	
	\onslide<+->
		42 overloadable operators + 4 unary forms
		
		Nicht überladbar: \verb|sizeof typeid . .* :: ?:|
\end{frame}


\begin{frame}[fragile]{Gruppiert nach Parameterzahl}
	\onslide<+->
		\begin{block}{Unäre Operatoren}
			\verb|! ++ --|~~sowie die unären Varianten von~~\verb|+ - * &|
		\end{block}
	
	\vspace{2em}
	
	\onslide<+->
		\begin{block}{Binäre Operatoren}
			(alle anderen)
		\end{block}
	
	\vspace{2em}
	
	\onslide<+->
		\begin{block}{Ternärer Operator}
			\verb|?:|
		\end{block}
\end{frame}




\subsection{Aufruf von Operatoren}

\begin{frame}[fragile]{Unäre Operatoren}
	\begin{block}{operator function invocation}
		For operator \cppop{@} and expression \cppop{a}:
		
		\vspace{1em}
		
		\begin{tabular}{l|l|l}
			expression	&	as member function	& as (global) function	\\
			\hline
			\cppop{@a}	&	\cppop{(a).operator@ ()}	&	\cppop{operator@ (a)}	\\
			\cppop{a@}	&	\cppop{(a).operator@ (0)}	&	\cppop{operator@ (a, 0)}	\\
			\cppop{a-\textgreater}	&	\cppop{(a).operator-\textgreater ()}	&	n/a \\
		\end{tabular}
	\end{block}
	
	\vspace{1em}
	
	\onslide<+->
		\lstinputlisting[linerange={1-6}]{cpp-code/operator-invocation.cpp}
\end{frame}


\begin{frame}[fragile]{Binäre Operatoren}
	\begin{block}{operator function invocation}
		For operator \cppop{@} and expression \cppop{a}:
		
		\vspace{1em}
		
		\begin{tabular}{l|l|l}
			expression	&	as member function	& as (global) function	\\
			\hline
			\cppop{a@b}	&	\cppop{(a).operator@ (b)}	&	\cppop{operator@ (a, b)}	\\
			\cppop{a=b}	&	\cppop{(a).operator= (b)}	&	n/a \\
			\cppop{a[b]}	&	\cppop{(a).operator[] (b)}	&	n/a \\
			\cppop{a(b)}	&	\cppop{(a).operator() (b)}	&	n/a \\
		\end{tabular}
	\end{block}
	
	\vspace{1em}
	
	\onslide<+->
		\lstinputlisting[linerange={8-12}]{cpp-code/operator-invocation.cpp}
\end{frame}




\subsection{Zweck von Operator-Überladung}

\begin{frame}[fragile]{Wozu Operator-Überladung?}
	\begin{itemize}
		\item intuitive Syntax z.B. für Matrizen \verb|A * B + C|
		\pause
		\item Hinteranderausführen, etwa \verb|a + b + c| statt \verb|add(add(a, b), c)| oder \verb|a.add(b).add(c)|
		\pause
		\item Syntax-Kompatibilität (z.B. Pointer und Smart-Pointer)
		\pause
		\item Spezialfälle
		\begin{itemize}
			\item address-of \cppop{\&}
			\item class member access \cppop{-\textgreater} 
			\item assigment \cppop{=}
		\end{itemize}
	\end{itemize}
\end{frame}




\subsection{Syntax}

\begin{frame}[fragile, b]{Einfache binäre Operatoren}
	\onslide*<+>
	{
		\begin{block}{\cppop{a.operator@ (b)}}
			\lstinputlisting[linerange={2-10}]{cpp-code/overloading-binary-operators.cpp}
		\end{block}
	}
	
	\onslide*<+>
	{
		Non-assigment operators und kein []
		
		\begin{block}{\cppop{operator@ (a, b)}}
			\lstinputlisting[linerange={14-22}]{cpp-code/overloading-binary-operators.cpp}
		\end{block}
	}
	
	\vspace{3em}
\end{frame}


\begin{frame}[fragile, b]{Binary assignment operators}
	\onslide*<+>
	{
		\begin{block}{\cppop{a.operator@ (b)}}
			\lstinputlisting[linerange={27-38}]{cpp-code/overloading-binary-operators.cpp}
		\end{block}
	}
	
	\onslide*<+>
	{
		Kein \cppop{[]} und kein \cppop{()}
		
		\begin{block}{\cppop{operator@ (a, b)}}
			\lstinputlisting[linerange={42-53}]{cpp-code/overloading-binary-operators.cpp}
		\end{block}
	}
	
	\vspace{1em}
\end{frame}


\begin{frame}{Unäre Operatoren: Präfix}
	\onslide*<+>
	{
		Non-assignment operators
		
		\begin{block}{\cppop{a.operator@ ()}}
			\lstinputlisting[linerange={2-9}]{cpp-code/overloading-unary-operators.cpp}
		\end{block}
	}
	
	\onslide*<+>
	{
		Non-assignment operators, kein -\textgreater
		
		\begin{block}{\cppop{operator@ (a)}}
			\lstinputlisting[linerange={13-20}]{cpp-code/overloading-unary-operators.cpp}
		\end{block}
	}
\end{frame}

\begin{frame}[fragile]{Unäre Operatoren: Suffix}
	Die einzigen unären Suffix-Operatoren sind \verb!++! und \verb!--!, also assigment-ops.
	
	\onslide*<+>
	{
		\begin{block}{\cppop{a.operator@ (0)}}
			\lstinputlisting[linerange={25-36}]{cpp-code/overloading-unary-operators.cpp}
		\end{block}
	}
	
	\onslide*<+>
	{
		\begin{block}{\cppop{operator@ (a)}}
			\lstinputlisting[linerange={40-51}]{cpp-code/overloading-unary-operators.cpp}
		\end{block}
	}
\end{frame}




\subsection{member functions oder non-member functions?}

\begin{frame}[fragile]{Fälle ohne Wahlmöglichkeit}
	Nur als member-function-Variante:~~\verb!= [] () ->!
	
	\vspace{1em}
	\pause
	
	Bei allen anderen Operatoren hat man die Wahl.
	
	Einschränkung: operator function invocation
	\begin{lstlisting}
		#include <ostream>
		#include <iostream>
		
		using namespace std;
		// definition of the cout object according to 27.3
		extern ostream cout;
		
		class MyClass { /* ... */ };
		std::ostream& operator <<(ostream& o, MyClass p)
		{ /* ... */ return o;}
		
		MyClass p;
		cout << p;	// calls: operator <<(cout, p)
	\end{lstlisting}
\end{frame}

\begin{frame}[fragile]{Fälle mit Wahlmöglichkeit}
	Habe keine guten Entscheidungshilfen gefunden!
	
	\vspace{1em}
	
	Bemerkung:\\
	Zwecks Symmetrie nimmt man meist die Variante mit der non-member function:
	\begin{lstlisting}
		class MyInt;
		MyInt operator+ (MyInt, int);
		MyInt operator+ (int, MyInt);
		
		class MyInt {
		    friend MyInt operator+ (MyInt, int);
		    // MyInt operator +(MyInt int);
		    friend MyInt operator+ (int, MyInt);
		    // n/a
		    
		private:
		    int m;
		};
	\end{lstlisting}
\end{frame}




\subsection{Wichtige Hinweise}

\begin{frame}[fragile]{Self-assignment}
	\onslide*<+>
	{
		\lstinputlisting[linerange={1-16, 37-38}]{cpp-code/self-assignment.cpp}
	}
		
	\onslide*<+>
	{
		\lstinputlisting[linerange={19-34, 37-38}]{cpp-code/self-assignment.cpp}
	}
\end{frame}

\begin{frame}[fragile]{Weitere Hinweise}
	\url{http://www.parashift.com/c++-faq-lite/operator-overloading.html#faq-13.9}
	
	\vspace{2em}
	
	\begin{itemize}[<+->]
		\item operator overloading ist nicht dafür da, dem class designer das Leben einfacher zu machen, sondern dem Nutzer einer Klasse!
		\item Operatoren sollten intuitive und eindeutige Bedeutung haben
		\item Identitäten sollten erhalten bleiben, z.B. \verb|x == x + y - y|
		\item manche Operatoren sollten den ersten Operanden verändern, z.B. \verb|= += *=|
		\item manche aber eben nicht, z.B. \verb|+ * /|
	\end{itemize}
\end{frame}
