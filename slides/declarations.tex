\section{forward declarations}

\begin{frame}[fragile]{Deklaration und Definition}
	\begin{block}{Standard, 3.1}
		Eine Deklaration führt einen Namen ein (in eine translation unit).
		Eine Deklaration ist zugleich eine Definition, außer:
		\begin{itemize}
			\item Bei Funktionen/Klassen: Fehlendem Rumpf \\
				{\tiny (Rumpf: das zwischen den geschweiften Klammern) }
			\item bei Variablen mit angegebenem linkage-specifier \verb|extern|
			\item Bei static member Variablen ohne initialisierung
		\end{itemize}
	\end{block}
	
	\vspace{1em}
	
	\uncover<+->
	{
		Eine Deklaration sagt: Es gibt einen Namen \\
		Eine Definition beschreibt, was genau mit dem Namen bezeichnet wird.
	}
\end{frame}


\begin{frame}{Deklaration und Definition: Beispiele}
	\onslide*<+>
		\lstinputlisting[basicstyle=\tiny]{cpp-code/decl-defn.cpp}
\end{frame}


\begin{frame}[fragile]{functions}
	\begin{block}{Voraussetzungen an Funktionen beim Aufruf (Standard, 3.2:3, 3.5:2)}
		\begin{itemize}
			\item Der Name muss zuvor deklariert sein.
			\item Es muss (irgendwo) eine entsprechende Definition geben.
		\end{itemize}
	\end{block}
	
	\begin{block}{Syntax}
		\begin{columns}[t]
			\column{0.5\textwidth}
				\emph{\ translation unit 1}
				\vspace{0.5em}
			\begin{lstlisting}
void foo(int i);
void bar() { foo(42); }
			\end{lstlisting}
			
			\column{0.5\textwidth}
				\emph{\ beliebige translation unit}
				\vspace{0.5em}
			\begin{lstlisting}
#include <iostream>
void foo(int i)
{
    std::cout << i;
}
			\end{lstlisting}
		\end{columns}
	\end{block}
\end{frame}


\begin{frame}[fragile]{Incomplete Types}
	Eine Klasse, die nur deklariert aber nicht definiert wurde, ist ein \emph{incomplete type}.
	Es dürfen keine »Dinge« mit incomplete type angelegt werden.
	Es dürfen sehr wohl Referenzen und Pointer auf incomplete types angelegt werden.
	
	Ein Pointer auf einen incomplete type ist modifizierbar, aber nicht dereferenzierbar.
	
	Bevor man einen Pointer/Referenz nutzt, sollte der type completed sein (definiert). Sonst darf man quasi nichts mit dem Pointer/Referenz tun. Siehe Standard, 3.2:4, 5.2.2:4, 5.3.1:4, 5.7:1, 8.3.5:6
	
	\begin{lstlisting}
		struct X;
		void foo(X); // alt: void foo(X x);
		
		struct X { int m; };
		void foo(X x)
		{
		    x.m = 42;
		}
	\end{lstlisting}
\end{frame}


\begin{frame}[fragile]{classes, types}
	\begin{block}{Syntax}
		\begin{columns}[t]
			\column{0.3\textwidth}
			\begin{lstlisting}
 class A;
struct B;
 union C;
  enum D;
			\end{lstlisting}
			
			\column{0.7\textwidth}
			\begin{lstlisting}
 class A { int m; };
struct B { void doit(); };
 union C { int m; char c; };
  enum D { X, Y, Z };
			\end{lstlisting}
		\end{columns}
	\end{block}
	
	\pause
	
	Wozu verwenden?
	\pause
	
	Für gegenseitige Abhängigkeiten oder auch information hiding!
	\begin{lstlisting}
class A;
class B;
class B { A* getA(); };
class A { void doB(B*); };
	\end{lstlisting}
\end{frame}

